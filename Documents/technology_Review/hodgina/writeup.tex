\documentclass[onecolumn, draftclsnofoot,10pt, compsoc]{IEEEtran}

\usepackage{graphicx}
\usepackage{url}
\usepackage{setspace}
\usepackage{geometry}
\usepackage{listings}
\usepackage{etoolbox}
\usepackage{pdflscape}

\patchcmd{\thebibliography}{\section*{\refname}}{}{}{}

\geometry{textheight=9.5in, textwidth=7in}

% 1. Fill in these details
\def \CapstoneTeamName{			              			 PlanteR-GB}
\def \CapstoneTeamNumber{					           			 Group 64}
\def \GroupMemberOne{				           				Austin Hodgin}
\def \CapstoneProjectName{	      	    Winter is Coming...}
\def \CapstoneSponsorCompany{		    Oregon State University}
\def \CapstoneSponsorPerson{		 			  				 Victor Hsu}

% 2. Uncomment the appropriate line below so that the document type works
\def \DocType{		%Problem Statement
				%Requirements Document
				Technology Review
				%Design Document
				%Progress Report
				}

\newcommand{\NameSigPair}[1]{\par
\makebox[2.75in][r]{#1} \hfil 	\makebox[3.25in]{\makebox[2.25in]{\hrulefill} \hfill		\makebox[.75in]{\hrulefill}}
\par\vspace{-12pt} \textit{\tiny\noindent
\makebox[2.75in]{} \hfil		\makebox[3.25in]{\makebox[2.25in][r]{Signature} \hfill	\makebox[.75in][r]{Date}}}}
% 3. If the document is not to be signed, uncomment the RENEWcommand below
\renewcommand{\NameSigPair}[1]{#1}

%%%%%%%%%%%%%%%%%%%%%%%%%%%%%%%%%%%%%%%
\begin{document}
\begin{titlepage}
    \pagenumbering{gobble}
    \begin{singlespace}
        \hfill
        \par\vspace{.2in}
        \centering
        \scshape{
            \huge CS Capstone \DocType \par
            {\large\today}\par
            \vspace{.5in}
            \textbf{\Huge\CapstoneProjectName}\par
            %\vfill
						\vspace{1in}
            {\large Prepared for}\par
            \Huge \CapstoneSponsorCompany\par
            \vspace{5pt}
            {\Large\NameSigPair{\CapstoneSponsorPerson}\par}
						\vspace{1in}
            {\large Prepared by}\par
						{\huge \CapstoneTeamNumber}\par
            \CapstoneTeamName\par
            \vspace{5pt}
            {
							\Large
							\NameSigPair{\GroupMemberOne}\par
            }
            \vspace{20pt}
        }
%\textbf{\textsuperscript{citation needed}}
				\newpage
        \begin{abstract}
					This is is where the abstract will be in the final draft
        \end{abstract}
    \end{singlespace}
\end{titlepage}
\newpage
\pagenumbering{arabic}
\tableofcontents
% 7. uncomment this (if applicable). Consider adding a page break.
%\listoffigures
%\listoftables
\clearpage
\singlespace
%References
		\subsection*{References}
		\begingroup
			\renewcommand{\addcontentsline}[3]{}% Remove functionality of \addcontentsline
			\renewcommand{\section}[2]{}% Remove functionality of \section
			%\cite[Sec 3.8]{sourceName}
			\bibliography{ref}
			\bibliographystyle{IEEEtran}
		\endgroup
		\newpage
		\newpage
	% Document body
	\section{Web Server Options}
		\subsection{Overview and Criteria}
		A hosted web interface is part of iteration 6 in the additional features
		section of our requirements document. As part of the requirement we will
		need a web server to serve the web page when a client sends a request.There three
		criteria that are needed for our web server for this project. The first is that
		the server will need to run on our controller that we chose. The second being
		that the server will need to be reliable on starting up and maintaining the
		server over long stretches of time. The third being memory and processor efficient.
		The controllers that we will be using will have limited memory and we will have
		other tasks running on the same controller so having a server that uses less
		memory and still maintains the features that we require is important.
		\subsection{Potential Choices }
		There is three different web servers that are under consideration. The first
		being Apache. Apache is a HTTP server project produced by Apache Software. The
		second web server is Lighttpd. Lighttpd is a web server that is an open source
		project. The third option is Niginx web service produced by NGINX Software.
		Below we will discuss each of these options and if they meet the criteria
		that is required for this project.
		\subsection{Apache Web Service}
		Apache web server that is developed by Apache Software. It is a widely used
		web server. Reviewing their about page \cite[pg 2]{Apache} they talk about
		building a stable and advanced web server that is robust, featureful, and
		free for anyone to use. This benefits us since we will not have to pay to use
		this server. Apache web server is also a process based server, this means
		that each connection gets its own thread on the CPU. This allows for if there
		is an error on one connection it does not bring down the whole server. This
		makes the server more reliable and stable. Apache is also for flexible configuration
		.htaccess files. It also comes standard on most Linux servers. Since it is
		one of the more popular options for web server it also has a lot of documentation.
		\subsection{Lighttpd Web Service}
		Lighttpd web server is a lightweight web server that is designed to be secure,
		fast, flexible and is optimized for high-performance. From their website \cite[pg 2]{Lighttpd}
		they state that they use non-blocking I/O event loop for running each single
		process that comes to the server. They talk about how the have a low memory
		footprint. Lighttpd is also an asynchronous server compared to Apache which is
		a synchronous server.
		\subsection{Nginx Web Service}
		Nginx is an open source web server developed by Nginx software. It is another
		asynchronous lightweight HTTP server. It differs from apache by that it has
		one master process to server requests. It then delegates tasks to worker
		proccess. This allows it to be very fast and efficient when serving static
		web pages. \cite[pg 2] {Nginx}
		\subsection{Discussion}
		Looking at all three options laid out above, Apache, Lighttpd, Nginx, they all
		handle serving web pages and processing request slightly differently. Apache for
		example uses a processes based approach. Meaning that each request gets its own
		tread on the CPU, this does require some overhead when it comes to resource.
		This is in contrast to both Lighttpd and Nginx web servers which are asynchronous
		servers are mainly event-driven and run on a single process and split into
		worker tasks from there to complete tasks. This makes them a little faster
		and  requires less total resources to run. It does have a problem where if
		a problem occurs in a connection it could bring the whole server down which
		would not happen to an Apache server since each connection is in its own thread.
		When looking at the criteria for this project, reliability, efficient use of
		resources, and ability to work on the chosen controller, they all meet these,
		though some do some better. For example, Apache is more reliable but does
		use more resources. This is in contrast to Lighttpd and Nginx which are both
		less reliable but are more efficient with their resources. They all can run
		on the raspberry pie which was chosen for the main control box.
		\subsection{Conclusion}
		In conclusion, looking at all three options we have chosen to go with the Apache web server.
		For this project we are looking to just host web pages and at most only a few connections
		to it will occur. One of the main criteria that we were worried about was that
		it would be stable once we had it working. Apache being one of the most stable
		web servers due to it using a process based server if a problem occurs with
		the connections then it won’t bring down the whole system. With that it meets
		all of our criteria for our project.
	\newpage
	\section{Mobile Options}
		\subsection{Overview and Criteria}
		Mobile interface is part of iteration eight in the additional feature section
		of our requirements document. As mobile computing becomes more and more popular
		it is important to take that into consideration when designing products today.
		As part of iteration eight of our project we would like to add this feature
		to our product. When choosing which option we would like to go with when
		incorporating mobile we have to look at three key criteria. The first being
		that it works on multiple mobile platforms, Android, and IOS. The second
		being that we would like it to be consistent with how the user interacts with the product.

		\subsection{Potential Choices }
		When looking at mobile options there are three main options to look at. The
		first being building a native mobile application for each platform. The second
		being building a mobile friendly web interface from iteration 6 of the project.
		The final one being building a  dedicated mobile site.
		\subsection{Native Mobile Application}
		building a native app for mobile platforms for users
		to download and install on their device. Building a native mobile
		app has several advantages and disadvantages when it comes to working with
		products. Below we will list the advantages and disadvantages of building a
		dedicated mobile application.
		\\\\
		\textbf{Advantages}:
		\begin{itemize}
			\item \textbf{Full Mobile Friendly}: This is designed and implemented with
			mobile in mind. Making it a more friendly mobile environment.
			\item \textbf{Quick Load Times} : With the app fully contained on the mobile
			it does not have pull a web page from a server.
		\end{itemize}

		\textbf{Disadvantages}:
		\begin{itemize}
			\item \textbf{Developing and Maintaining Multiple Platforms}: With designing
			and creating a native mobile app you have to create it on two different platforms.
			This can be time consuming.
		\end{itemize}
		\subsection{Build Mobile Friendly Website}
		Wth building a mobile friendly website we would develop the site with mobile
		in mind. Building a mobile friendly website has several advantages and disadvantages
		when working with products. Below are listed the advantages and disadvantages
		for building a mobile friendly website.
		\textbf{Advantages}:
		\begin{itemize}
			\item \textbf{Consistent With Desktop Content}: This would allow the desktop
			and mobile environments to feel consistent and not make users learn two
			different interfaces
			\item \textbf{Less To Maintain}: You will only need to build and maintain
			one site making it easier to make changes to both.
		\end{itemize}
		\textbf{Disadvantages}:
		\begin{itemize}
			\item \textbf{General Usability}: Changing settings or checking the same
			information might be more difficult on a mobile platform due to how they go
			through the pages.
			\item \textbf{Slower Loading Times}: With the the amount of information pushed
			 back and forth between the device and could be slower due to it going over
			 3g or 4g speeds.
		\end{itemize}

		\subsection{Build Mobile Dedicated  Site}
		Building a dedicated mobile app would be creating the site again but more
		mobile friendly and delivering that to mobile devices instead of the desktop
		site. Below are the advantages and disadvantages of creating a mobile dedicated site.

		\textbf{Advantages}:
		\begin{itemize}
			\item \textbf{Mobile Focused}: With building a mobile dedicated site it allows
			you to make it more mobile friendly.
			\item \textbf{Consistent Style}: It allows for you to tweak the style of the
			main site to allow for it to be more mobile friendly but still keep its look
			and feel.
		\end{itemize}

		\textbf{Disadvantages}:
		\begin{itemize}
			\item \textbf{Increased Maintains Time/Cost}: With having two sites to maintain,
			you will have to put more time and energy creating and maintaining it.
			\item \textbf{Slower Service}: Since the server has to check what type of
			device that is requesting the page it takes some processing time.
		\end{itemize}
		\subsection{Discussion}
		Looking at all three options they all have advantages and disadvantages. For
		our project we are looking at a few key criteria and those are that it has to
		work on multiple platforms and that it needs to be consistent with the desktop
		environment. All three can easily be made to be consistent with the desktop
		environment. With building native application for mobile that would require
		applications developed for both Android and IOS. This is time consuming and
		difficult to maintain as a change in one would require a change in the other.
		This is in contrast to the other two options, where both of them will work on any device
		given that it has a browser. Looking at the other two options they both have very
		similar advantages and disadvantages. The main disadvantage to building a dedicated
		mobile site is that you will have to build and maintain two sites which compared to
		building a mobile friendly website where you just have to maintain one site.
		They both meet both criteria we have for our mobile options and are both strong
		options.
		\subsection{Conclusion}
		Between the three options, building a native mobile application, building a
		mobile dedicated site and building a mobile friendly site we choose to go with
		building a mobile friendly site. With keeping mobile devices in mind from the
		beginning of the web interface it makes it easier to develop and maintain over
		time. This also cuts down on development time over all as we only need to
		create one web site.


	\section{Moisture Sensors}
		\subsection{Overview and Criteria}
		Adding  moisture sensors is part of iteration 10 in additional features section
		of the requirements document. We would like to be able to add moisture sensors
		to the enclosure to monitor the moisture of the soil the plants are in to
		get an idea of when you should water your plants. To do this we have several
		criteria that we have to look at. The first being that it must be able to
		interact with our chosen microcontroller. The second being that it must
		read the moisture of the soil.
		\subsection{Potential Choices }
		There are three different choices that we could go with. The first is going
		with an Octopus soil moisture sensor from Elecfreaks \cite[pg 2] {Octopus_soil_sensor},
		We could design and custom print our own moisture sensors or we could go with
		a  SparkFun Soil Moisture Sensor from SparkFun \cite[pg 2] {SparkFun_Soil_Moisture_Sensor}.
		\subsection{Octopus Soil Moisture Sensor}
		Octopus soil moisture sensor is a moisture sensor that can be purchased trhough
		Elecfreaks. This sensor is a lot like other moisture sensors in shape and how
		it connects to a micro controller. The way this moisture sensor is unique is
		that it can connect to the sensor modules. This can connect to any arduino
		based microcontroller.
		\subsection{Design and Print Own Sensors}
		One option would be to design our own sensor using a program like Eagle and
		have it printed and sent to use. The advantage to this is that we can make the
		sensor exactly how want it in the shape that would fit in our enclosure. This
		would also be ideal if we need a bunch of them. The disadvantage for this would
		that we would have to spend time designing it and then have to wait for them
		to print. This would only be ideal if we were going through to a production
		unit.
		\subsection{SparkFun Soil Moisture Sensor}
		SparkFun soil moisture sensor is a moisture sensor that can be purchased through
		Sparkfun. It can be used with any arduino microcontroller. This allows us to be able
		monitor the soil moisture. The SparkFun soil moisture sensor is a lot like
		many other sensors in design.
		\subsection{Discussion}
		Moisture sensors are simple sensors so most of them are similar in shape,
		size and how they read in moisture. For this reason it is possible to design
		our own and get it the shape and size that we would like. The disadvantage of this
		is that we would have to design it and then have them printed and they are
		not guaranteed to work. This is why the other two options were chosen. The
		Octopus Soil Moisture Sensor and the Sparkfun Soil Moisture Sensor are very similar
		The Octopus soil moisture sensor has the ability to connect to other sensors.
		This is a nice feature if we would like to add any other sensors. The SparkFun
		Soil moisture sensor doesn't have any special feature other than reading in
		the moisture level. Though it is more expensive than the Octopus moisture sensor.
		\subsection{Conclusion}
		Looking at all the options it seems like the Octopus soil moisture sensor is
		the option we will be going with. Designing and having it printed would be
		time consuming and expensive up front. The SparkFun soil moisture sensor is
		has all the features that we need but is more expensive than the Octopus soil
		moisture sensor making it the sensor that we will use for our project.

\end{document}
