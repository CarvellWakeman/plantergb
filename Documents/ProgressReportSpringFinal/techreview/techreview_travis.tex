% Document body
\section{Microcontroller}
	\subsection{Overview and Criteria}
	The microcontroller will run parts or all of our software. With this in mind,
	we need to find a controller that fits within certain criteria. The
	controller must
	\begin{itemize}
		\item Controller is reprogramable
		\item Controller contains I/O pins
	\end{itemize}
	The controller must be reprogramable as to allow the user to change files;
	This also facilitates testing. The I/O pins will allow the LED strips to
	interface with the controller, as well as allow additional features to be
	added.

	\vspace{5mm}
	\noindent Following these criteria, we have a few additional features that would be
	benifical. These include:
	\begin{itemize}
		\item Controller allows multiple applications to run simultaneously
		\item Controller allows connection via a wireless network
		\item Controller is capable of hosting a web page
	\end{itemize}
	Allowing the controller to run multiple applications will facilitate
	multiple features to run on a single controller; For example the controller
	will run the LED interface, as well as hosting a web page. Allowing
	connection via a wireless network will help with additional features on a
	single controller.
	\subsection{Potential Choices}
	During our research we have found three commonly used microcontrollers that
	fit our criteria. The first is the Teensy 3.2. The second is the Arduino Uno,
	and the third is the Raspberry Pi Zero W. These three were chosen for their
	ease of use, price, and documentation.
	\subsection{Teensy 3.2}
	The Teensy 3.2 is a commonly used microntroller for small scale LED projects,
	using a cortex-m4 ARM processor running at 72MHz \cite[Pg 7]{K20}. Storage consists
	of 256KB of flash memory that will contain the Arduino or C flashed to it.
	The Teensy has 34 I/O pins, and is powered by 5V supplied by a micro USB
	cable.

	\vspace{5mm}
	\noindent The Teensy 3.2 is a fairly small board, being only 1.4 inches long
	by 0.7 inches wide, which is great for projects where space is limited. The
	Teensy 3.2 is priced at \$20.

	\vspace{5mm}
	\noindent Advantages of the Teensy 3.2:
	\begin{itemize}
		\item Small form factor
		\item Runs Arduino and C programs
		\item Compatible with many LED libraries
	\end{itemize}
	Disadvantages of the Teensy 3.2:
	\begin{itemize}
		\item Not robust enough for much more than the LEDs
		\item Documentation is lacking with regards to LEDs
	\end{itemize}
	\subsection{Arduino Uno}

	\vspace{5mm}
	\noindent The Arduino Uno is a popluar microcontroller for many small electronics
	projects. The Uno contains an ATmega328 controller running at 20 million
	instructions per second (MIPS) at 20MHz with 32KB of flash
	memory\cite[Pg 7]{atmel}. Containing 14 digital I/O pins of which 6
	can be used for PWM outputs, and six analog inputs \cite[Pg 7]{arduino}.
	The Uno uses an input voltage of 5v off a barel jack. The Uno has a large community with
	a lot of documentation.

	\vspace{5mm}
	\noindent The Arduino Uno is 2.7 inches by 2.1 inches making it more ideal for
	projects where size isnt a limiting factor.

	\vspace{5mm}
	Advantages of the Arduino Uno:
	\begin{itemize}
		\item Compatible with many LED libraries
		\item Addon shields allow for further features to be added
		\item Interfacing with the LEDs is easiest with this device
		\item Documentation is fantastic
	\end{itemize}
	Disadvantages of the Arduino Uno:
	\begin{itemize}
		\item Shields sold seperatly
		\item Not robust enough to run more than LEDs and sensors
		\item Wireless connections require more hardware.
	\end{itemize}
	\subsection{Raspberry pi Zero W}

	\vspace{5mm}
	\noindent The Raspberry Pi Zero W is more of a microcomputer than a microcontroller.
	It is capable of running a full linux operating system and
	contains an ARM processory at 1GHz with 512MB of RAM making this the most
	powerful option we looked at. This Pi uses 40 GPIO pins for
	external interface, uses a micro SD card for storage, and supports digital
	video out via a mini HDMI\cite[Pg 7]{pizero}.

	\vspace{5mm}
	\noindent Additionally, the Pi Zero W contains a wireless and bluetooth chip built in.
	The Pi Zero W is another small controller being 2.5 inch by 1.2 inch
	making this controller another great choice for smaller projects.
	Raspberry Pi Zero W is priced at \$10, with a huge community and plenty of documentation.

	\vspace{5mm}
	\noindent While the Raspberry Pi can run many RGB LEDs, specific models use internal
	clock timings to time the LED signals. This makes running these on the Pi,
	which uses a multi-tasking linux operating system, a pain.

	\vspace{5mm}
	\noindent Advantages of the Raspberry Pi Zero W:
	\begin{itemize}
		\item Can host local web pages
		\item Can run a light linux operating system
		\item Great documentation
	\end{itemize}
	Disadvantages of the Raspberry Pi Zero W:
	\begin{itemize}
		\item Multitasking operating system make clock based LEDs a pain to
		work with.
	\end{itemize}

	\subsection{Discussion}
	\noindent Each of these controllers meet our
	requirements. The Teensy, being our least powerful device, is a great choice
	for small LED projects. It appears the Teensy will not be able to handle
	both setting the LEDs and changing the time the LEDs will turn on and off
	at the same time. This makes the Teensy a great option for only a part of
	the project.

	\vspace{5mm}
	\noindent The Arduino Uno, being our second most powerful device on the
	list, is capable of handling both setting the LEDs and changing the time
	the LEDs will turn on and off. This makes it a great option for this
	project. Since it is an Arduino, we are limited to two coding languages;
	Arduino, and C. The only way for the Uno to connect to a wireless network,
	is by using an external shield or USB wireless adapter. This means we will
	need additional hardware to connect to any network.


	\vspace{5mm}
	\noindent The Raspberry Pi Zero W is the most powerful controller option on
	this list. This gives us more options in terms of additional
	features. With its great support, amount of GPIO pins, and ability to run
	several applications at the same time, this is would be a great option for
	a sort of "master" running our some of our additional features such as a
	local web page, and moisture sensors, while another controller can handle
	the LED controls.

	\subsection{Conclusion}
	\noindent Given our criteria, and our additional features, we have decided
	to use the Raspberry Pi Zero W and a Teensy 3.2. Because of the power of the
	Pi Zero W, the ability to run a full linux operating system, price, size,
	and number of GPIO pins, this controller fits everything we need and more.
	The LED libraries make using the Teensy as the LED driver much more
	attractive.LED Libraries will be discussed in another section. The Pi Zero
	W's ability to host web pages, and connect to wireless networks with no a
	dditional hardware, we can achieve more of our additional features.

\section{LEDs}
	\subsection{Overview and Criteria}
	\noindent The LEDs are the centeral point of this project. As such, we need
	to pick a type that suits our needs. Our criteria are:
	\begin{itemize}
		\item The LEDs must be RGB
		\item The LEDs strips must be modular
	\end{itemize}
	\noindent The LEDs we choose have to be RGB, that is the point of this
	project. With modular LEDs, we will be able to create any size kit we want,
	depending on how many LEDs we want to use.
	\subsection{Potential Choices}
	\noindent We have three great choices for LEDs. The first is the Adafruit
	WS2812 Neopixel LED strip. The second is the DOTSTAR APA102 LED strip.
	Lastly we have the WS2801 diffused LED strand. Each of these fits all of
	our criteria.
	\subsection{Neopixel WS2812}
	\noindent The WS2812 Neopixel LEDs from Adafruit is popular for many LED
	projects. These LEDs are digitally addressable, meaning we can set each LED
	color individualy; Each LED has a shift-registers, chained throughout the
	strip which allows us to shorten the strip, or add more to the end
	\cite[Pg 7]{neo}. Once you set the color, you can disconnect the strip from
	the controller, and as long as its still being powered, it will remain
	thanks to the build in PWM into each LED-chip. Powering the LEDs comes
	from solder pads on the side of each LED that will provide 5V and up to 2A,
	ground, and a data line.


	\vspace{5mm}
	\noindent The Neopixel WS2812 LEDs come with 60 LEDs across a meter for \$24.99.

	\vspace{5mm}
	\noindent Advantages of the Neopixel WS2812:
	\begin{itemize}
		\item Documentation is great
		\item Digitally addressable
		\item Modular
		\item Many libraries
	\end{itemize}
	Disadvantages of the Neopixel WS2812:
	\begin{itemize}
		\item Clock controlled
	\end{itemize}
	\subsection{DOTSTAR APA102}
	\noindent The DOTSTR APA102 are an alternative to adafruits Neopixel LEDs. Instead
	of working on a single data pin, these work on a 2-wire SPI, meaning data
	can traverse the strip much faster than the Neopixel PWM system. These do
	not require timing meaning clock cycles will not affect these
	\cite[Pg2]{dotstar}. These contain 30 LEDs per meter with 24-bit color,
	8 bits for each red, green, and blue. Like the Neopixel, each LED acts Like
	a shift register, which means we can, if not hardware limited, control an
	infinite number of LEDs. The full meter costs \$19.95.

	\vspace{5mm}
	\noindent Advantages of the Dotstar APA102:
	\begin{itemize}
		\item SPI allows data to travel faster down the LED strip.
	\end{itemize}
	Disadvantages of the Dotstar APA102:
	\begin{itemize}
		\item Less LED per meter than Neopixel
		\item SPI can be hard to set up
		\item Documentation is lacking
	\end{itemize}
	\subsection{WS2801 Diffused LED Strand}
	\noindent The WS2801 LED strand runs similarly to the Neopixel LEDS as they
	are clock based. Instead of a strip, they are in a strand of weatherproof
	"dots" that fit through a 12mm hole\cite[Pg 7]{strand}. These are diffused
	LEDs meaning the light is spread more evenly, reducing hard edges and
	shadows. Like the other two options, these are 24-bit colors. Similarly to
	the Neopixel's thes are PWM driven and must be clocked by the controller.

	\vspace{5mm}
	\noindent These only come in strands of 25, making these not easily modular without
	modifications. They are run off of 5V at up to 2A, just like the previous
	choices and are priced at \$39.95.

	\vspace{5mm}
	\noindent Advantages of the WS2801 Diffused LED Strands:
	\begin{itemize}
		\item Diffused lighting helps reduce hard shadows
		\item Dot based LED gives more options for hiding wiring
		\item Many libraries
	\end{itemize}
	Disadvantages of the WS2801 Diffused LED Strands:
	\begin{itemize}
		\item Not easily modular
		\item clock controlled
		\item Forced to use 25 pixels
	\end{itemize}
	\subsection{Discussion}
	The WS2812 and the WS2801 use the same clock based control, making the only
	difference is the style of LED. Both the WS2812 and the DOTSTAR LEDs are
	strips of flat LEDs, with the Dotstar being about \$5 cheaper.

	\vspace{5mm}
	\noindent The DOTSTAR uses an SPI connection from the making it an interesting choice.
	The major complaint is the lack documentation.
	\subsection{Conclusion}
	\noindent Because of the documentation, the libraries supported, and size,
	we have chosen to go with the WS812 Neopixel LEDs. These are the LEDs we
	have seen for many different projects, has great documentation from both
	Adafruit, and users, is supported by many libraries, and can be split up
	into smaller strips if needed, makes this our best choice.


\section{LED Libraries}
	\subsection{Overview and Criteria}
	\noindent LED libraries are what the microcontroller uses to interface with
	the LEDs. We need to pick a library that fits with our microcontroller, and
	LED choice. It is important to note that there exists libraries to run
	clock based LEDs on the raspberry pi using hardware PWM. These, however
	are created by other users, and are not updated often. Our
	criteria are:
	\begin{itemize}
		\item Library must work with either the Raspberry Pi, or Arduino Uno
		\item Library must work with the WS2812 Neopixel LEDs
	\end{itemize}
	\noindent We need to be sure the library we want to use is compatible with
	the controller we are using, otherwise the whole project will not work.
	Similarly, we need to make sure the library can work for the LEDs of our
	choice.
	\subsection{Potential Choices}
	\noindent For libraries, we have several choices. The first is fastLED. The
	second is the Adafruits Neopixel library, and the third choice is to build
	our own library. Each of these options will fit our criteria.
	\subsection{FastLED}
	\noindent FastLED is an arduino based library created to make programming
	LEDs faster and easier. This library has support for both SPI, and 3-wire
	chipsets\cite[Pg 7]{fastLED}. Many projects found using the Neopixel LEDs
	used this library.

	\vspace{5mm}
	\noindent This library is written in C++ and can be used in for anything
	that can run Arduino, such as the Arduino Uno and Teensy 3.2. Supporting
	both SPI and 3-wire chipsets allow for many different types of LEDs to
	interface, however it cannont be used on multi-tasking devices such as the
	raspberry pi. This is mostly due to the LEDs supported have a problem with
	clock timing, with the exception of LEDs such as the APA102.

	\vspace{5mm}
	\noindent Advantages to FastLED:
	\begin{itemize}
		\item Supports many different LED structures
		\item Great documentation
	\end{itemize}
	Disadvantages to FastLED:
	\begin{itemize}
		\item limited to running on Arduino
	\end{itemize}
	\subsection{Adafruit Library}
	\noindent The Adafruit Neopixel Library is specifically designed for their Neopixel
	LEDs. This library is written in C++ and is made for use for Arduino
	projects. That means this can be run on anything that can run Arduino such
	as the Arduino Uno and the Teensy 3.2. Adafruit has an article discussing
	the use of the library\cite[Pg 7]{neolib}.


	\vspace{5mm}
	\noindent Past this, the library has little in the way of documentation. The article
	discusses basic use, but doesn't go into much past their test script. This
	will lead to more time spent digging into the code base. In our research we
	have found many different projects using this library.

	\vspace{5mm}
	\noindent Advantages of the Adafruit Neopixel Library:
	\begin{itemize}
		\item Built specifically for Neopixel
		\item Built for use on Arduino devices
	\end{itemize}
	Disadvantages of the Adafruit Library:
	\begin{itemize}
		\item Can only use Neopixel LEDs
		\item Documentation is a little lacking
	\end{itemize}
	\subsection{Build our Own}
	\noindent Building our own library will reduce a lot of our problems of other
	libraries. We can code specifically for the LED and controller we want.
	This reduces the amount of added code the controller will use. Additionally,
	we will have better knowledge of how to write the project, as we are the
	ones who wrote it.

	\vspace{5mm}
	\noindent The major problem that comes with building our own LED library is time.
	Havig to write our own library will require extensive resarch, writing, and
	testing time that can detract from our original project.
	\subsection{Discussion}
	\noindent FastLED is a very popular LED library used in many projects we found
	during our research. With its broad LED support, and great documentation,
	it is an appealing option. The Neopixel library, while only made for use on
	the Neopixel LEDs, isn't bloated by the addition of these other LED types.
	Building our own library would allow us to intergrate only features we need
	for the specific LED type we are using. This however, will require more
	time for us to complete.
	\subsection{Conclusion}
	\noindent Because of the documentation and LED support, we have decided to use
	FastLED as our LED library. FastLED has great support, and allows many LED
	types for future expansion if needed.
