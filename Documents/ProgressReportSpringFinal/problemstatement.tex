	\section*{Problem Description}
	During the cold and dark winter months, plant growth becomes difficult at best, and calamitous at worst.
	In our state of Oregon we are well accustomed to overcast, low temperatures, and rain throughout the winter, spring, and fall months.
	Our client has a desire to grow herbs and plants indoors during these long, cold months. Solving this problem is a desire held by all members of our team.
	Much like our client, we enjoy cooking with fresh herbs and vegetables even through the winter.
	\\\\Many plants and herbs such as tomatoes, basil, ferns, and plants from foreign climates have an ideal set of environmental conditions that allow for optimum growth.
	Variables such as light wavelength, intensity, temperature, moisture, and pests all have an impact on a plant's health and yield.
	These factors lead many growers to bring their plants indoors during the winter months.
	\\\\Even in a climate controlled dwelling, the problem persists. Humans and plants require very different living conditions.
	Residents come and go from work, school, and vacation while plants have evolved to expect the 24hr day-night cycle.
	Busy humans can leave plants neglected or without light for days. Current interior plant lighting systems can be expensive and offer little to no customization.
	Plant growth systems that do not adapt to both the requirements of their plants and their gardener's schedule lead to frustration, low plant yield, and plant death.

	\section*{Proposed Solution}
	The client's primary concern is solving the problem of interior plant lighting.
	The features required to do this are clearly defined in the \textit{required features} section, but specific implementation details were left to our team.
	Each of us have skills in an area related to this project, and together we have the ability to go above and beyond what was requested by the client.
	We have built the set of \textit{additional features} listed below that we believe compliment the client's vision for this project.
	\\\\ Given proper design and documentation, the additional features are likely to be completed barring any complications.
	Features that are especially difficult to accomplish, unlikely to be completed, or are outside the primary scope of the project are listed as \textit{stretch goals}.
