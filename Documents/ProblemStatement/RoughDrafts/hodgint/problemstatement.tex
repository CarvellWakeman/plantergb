\documentclass[onecolumn, draftclsnofoot,10pt, compsoc]{article}
\usepackage{graphicx}
\usepackage{url}
\usepackage{setspace}

\usepackage{geometry}
\geometry{textheight=9.5in, textwidth=7in}

% 1. Fill in these details
\def \CapstoneTeamName{		The Party Parrots}
\def \CapstoneTeamNumber{		64}
\def \GroupMemberOne{			Austin Hodgin}
\def \GroupMemberTwo{			Max Schmidt}
\def \GroupMemberThree{			Travis Hodgin}
\def \GroupMemberFour{			Zach Lerew}
\def \CapstoneProjectName{		Winter is coming...}
\def \CapstoneSponsorCompany{	Oregon State University}
\def \CapstoneSponsorPerson{		Victor Hsu}

% 2. Uncomment the appropriate line below so that the document type works
\def \DocType{		Problem Statement
				%Requirements Document
				%Technology Review
				%Design Document
				%Progress Report
				}
			
\newcommand{\NameSigPair}[1]{\par
\makebox[2.75in][r]{#1} \hfil 	\makebox[3.25in]{\makebox[2.25in]{\hrulefill} \hfill		\makebox[.75in]{\hrulefill}}
\par\vspace{-12pt} \textit{\tiny\noindent
\makebox[2.75in]{} \hfil		\makebox[3.25in]{\makebox[2.25in][r]{Signature} \hfill	\makebox[.75in][r]{Date}}}}
% 3. If the document is not to be signed, uncomment the RENEWcommand below
\renewcommand{\NameSigPair}[1]{#1}

%%%%%%%%%%%%%%%%%%%%%%%%%%%%%%%%%%%%%%%
\begin{document}
\begin{titlepage}
    \pagenumbering{gobble}
    \begin{singlespace}
    	%\includegraphics[height=4cm]{coe_v_spot1}
        \hfill 
        % 4. If you have a logo, use this includegraphics command to put it on the coversheet.
        %\includegraphics[height=4cm]{CompanyLogo}   
        \par\vspace{.2in}
        \centering
        \scshape{
            \huge CS Capstone \DocType \par
            {\large\today}\par
            \vspace{.5in}
            \textbf{\Huge\CapstoneProjectName}\par
            \vfill
            {\large Prepared for}\par
            \Huge \CapstoneSponsorCompany\par
            \vspace{5pt}
            {\Large\NameSigPair{\CapstoneSponsorPerson}\par}
            {\large Prepared by }\par
            Group\CapstoneTeamNumber\par
            % 5. comment out the line below this one if you do not wish to name your team
           % \CapstoneTeamName\par 
            \vspace{5pt}
            {\Large
                \NameSigPair{\GroupMemberOne}\par
                \NameSigPair{\GroupMemberTwo}\par
                \NameSigPair{\GroupMemberThree}\par
		 \NameSigPair{\GroupMemberFour}\par
            }
            \vspace{20pt}
        }
        \begin{abstract}
        % 6. Fill in your abstract    
This senior capstone project involves a system of lights controlled by a micro controller to aid in the growth of herbs indoors. Winter months impede the growth of many herbs and other plants, making growing outdoors during this time difficult, if not impossible. Interior plant growing can be a pain as well. Existing plant lighting system can be difficult to use, contain little to no customization options and often only use a single light color and intensity and a set on, off cycle, if it has one. Research has shown that plants thrive under more specific shades and intensities of light. This project will allow users to control the color, and intensity of LED strips as well as control the times they turn on, and off. More control over these conditions will allow for high yields, as well as healthier plants. The controller will  allow users to define set times for the lights to turn on, and turn off, change of colors, and intensity. Giving control of these aspects will allow users to have a more efficient indoor plant growing setup. 
        \end{abstract}     
    \end{singlespace}
\end{titlepage}
\newpage
\pagenumbering{arabic}
\tableofcontents
% 7. uncomment this (if applicable). Consider adding a page break.
%\listoffigures
%\listoftables
\clearpage

% 8. now you write!
\section{Problem Description}
Growing outdoors during the cold and dark winter months can be difficult, if not impossible. Winter brings cold air, and dark days that can be destructive to many plants. Oregon is a prime example of this. Oregon is accustomed to cloudy, overcast days, and intense cold. Plants and herbs such as basil, ferns, and plants from other climate type a required environment in order for them to grow. These include temperature, moisture, light color and intensity. These factors lead many experienced growers to bring their plants indoors during this month.
\\
\\
Even with common indoor growing kits, these plants can still be affected. Human environments rarely act like the environments these plants need. Current grow light setups can be expensive, or lack features to combat this issue. Lights staying on too long, and light color can still be an issue. Residents leave for school, work, or go on vacation, leaving these plants neglected. These lighting systems lack one major feature, the ability to customize. The ability to set on and off times, and color and intensity of light is key to an effective indoor growing system.
\\
\\
This is where our client, Victor Hsu, our client, wants to help. Victor has a desire for a more sophisticated indoor growing system than what is already on the market. This is a desire that each of us hold. We all enjoy cooking, and using fresh herbs are always ideal. 

\section{Proposed Solution}
Our team includes four students ,each with our own skills in areas related to the project. Together we have the ability to go beyond what was requested by the client. The required features, shown below, are clear and straightforward are what is requested by the client. The client's first concern is creating an interior plant lighting system using RGB LED's controlled by a micro controller such as an arduino. Our team, however, believes we have enough skills and experience to create something bigger. With proper design and documentation, we believe adding more to this project will make a big difference. These additional features are shown in the Additional Features section. Further additions made that are unlikely to be completed in the time provided, outside the scope of the project, or are difficult to accomplish are listed as stretch goals.

\subsection{Required Features  -V1.0}
\begin{itemize}
	\item Single plant bed with RGB lighting
	\begin{itemize}
		\item A strip of re-programmable LEDs with a single controller on a single planter
		\item Service running on a micro controller that can control the power state of LED lights that are attached
		\item Configuration file that contains settings for light state.
		\item Changes to the configuration file are recognized and applied by the controller
	\end{itemize}
\end{itemize}
\begin{itemize}
	\item User based light variation
	\begin{itemize}
		\item Light state (color, intensity), specified by user in a weekly or daily schedule
	\end{itemize}
\end{itemize}
\begin{itemize}
	\item individual control over multiple strips or zones
	\begin{itemize}
		\item Controller supports individual control over multiple strips
		\item Data pin controls a chain of light strips
	\end{itemize}
\end{itemize}
\begin{itemize}
	\item Simple user interface for basic control
	\begin{itemize}
		\item Need a way to control the system
		\item Interface to edit configuration settings and allow the transfer of new settings to the controller
		\item All settings can be changed from this interface, although it may not be user friendly
	\end{itemize}
\end{itemize}

\subsection{Additional Features -V2.0}
\begin{itemize}
	\item Web Server and local network client interface for easier control
	\begin{itemize}
		\item Hosted LAN web interface
		\item Easy to use and responsive interface
		\item System settings can be updated over network without need for physical access
	\end{itemize}
\end{itemize}
\begin{itemize}
	\item Different colors on a single LED strip
	\begin{itemize}
		\item Multiple colors and intensities on a single LED strip using indexing
	\end{itemize}
\end{itemize}
\begin{itemize}
	\item Flexible zoning and sub-zoning
	\begin{itemize}
		\item Individual strip groups can be zoned and controlled together
	\end{itemize}
\end{itemize}
\begin{itemize}
	\item Android wrapper app for web server
	\begin{itemize}
		\item Mobile support for web server
		\item Android application acting as a wrapper
	\end{itemize}
\end{itemize}
\begin{itemize}
	\item Custom enclosure with vertical lighting
	\begin{itemize}
		\item Custom designed planter that fits controller and LED strips
	\end{itemize}
\end{itemize}
\begin{itemize}
	\item Humidity and temperature monitoring
	\begin{itemize}
		\item Control over additional humidity and temperature hardware
		\item Web interface plug in to monitor humidity and temperature
	\end{itemize}
\end{itemize}

\subsection{Stretch Goals -V3.0}
\begin{itemize}
	\item Modular light strip
	\begin{itemize}
		\item Light strips that attach together easily
		\item Automatic detection of lighting setup for easier indexing on user end
	\end{itemize}
\end{itemize}
\begin{itemize}
	\item Learning tools built into interface
	\begin{itemize}
		\item Interface that provides tips and suggestions to improve growing and gardening
	\end{itemize}
\end{itemize}
\begin{itemize}
	\item Modular planting enclosure
	\begin{itemize}
		\item self contained enclosure with plug-and-play connectors for automatic setup of multiple planter
	\end{itemize}
\end{itemize}
\begin{itemize}
	\item Watering system
	\begin{itemize}
		\item Automatic or scheduled water. Requires costume enclosure
	\end{itemize}
\end{itemize}
\begin{itemize}
	\item Communication for multiple grow light systems separated by distance
	\begin{itemize}
		\item Wireless system to allow control over multiple grow light systems with multiple controllers
	\end{itemize} 
\end{itemize}

\section{Performance Metric}
Performance metrics have been a topic of dispute with this team. The client's requirements are easily manageable for our team of four people. Our internal expectations are much higher. In this case, success should be defined by the set of features needed to make the system usable. Our clients goals define the system, however, there is room to expand. Below are the features that are necessary to define the project as a success. Extra features may be added if time allows.
\begin{itemize}
	\item \textbf{All features described by the client}, listed in the Required features section, are complete
	\item An alternative interface as described in the Additional features section. An interface allows easier user control over settings without physical access to the controller
	\item Interface will likely be a web server hosted on the controller, but may change
\end{itemize}

\end{document}